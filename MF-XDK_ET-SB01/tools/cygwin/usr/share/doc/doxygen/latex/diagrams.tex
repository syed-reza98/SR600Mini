Doxygen has built-\/in support to generate inheritance diagrams for C++ classes.

Doxygen can use the \char`\"{}dot\char`\"{} tool from graphviz to generate more advanced diagrams and graphs. Graphviz is an open-\/source, cross-\/platform graph drawing toolkit and can be found at \href{http://www.graphviz.org/}{\tt http://www.graphviz.org/}

If you have the \char`\"{}dot\char`\"{} tool in the path, you can set \hyperlink{config_cfg_have_dot}{HAVE\_\-DOT} to {\ttfamily YES} in the configuration file to let doxygen use it.

Doxygen uses the \char`\"{}dot\char`\"{} tool to generate the following graphs: 
\begin{DoxyItemize}
\item A graphical representation of the class hierarchy will be drawn, along with the textual one. Currently this feature is supported for HTML only.\par
 {\bfseries Warning:} When you have a very large class hierarchy where many classes derive from a common base class, the resulting image may become too big to handle for some browsers. 
\item An inheritance graph will be generated for each documented class showing the direct and indirect inheritance relations. This disables the generation of the built-\/in class inheritance diagrams. 
\item An include dependency graph is generated for each documented file that includes at least one other file. This feature is currently supported for HTML and RTF only. 
\item An inverse include dependency graph is also generated showing for a (header) file, which other files include it. 
\item A graph is drawn for each documented class and struct that shows: 
\begin{DoxyItemize}
\item the inheritance relations with base classes. 
\item the usage relations with other structs and classes (e.g. class {\ttfamily A} has a member variable {\ttfamily m\_\-a} of type class {\ttfamily B}, then {\ttfamily A} has an arrow to {\ttfamily B} with {\ttfamily m\_\-a} as label). 
\end{DoxyItemize}
\item if \hyperlink{config_cfg_call_graph}{CALL\_\-GRAPH} is set to YES, a graphical call graph is drawn for each function showing the functions that the function directly or indirectly calls. 
\item if \hyperlink{config_cfg_caller_graph}{CALLER\_\-GRAPH} is set to YES, a graphical caller graph is drawn for each function showing the functions that the function is directly or indirectly called by. 
\end{DoxyItemize}

Using a \hyperlink{customize}{layout file} you can determine which of the graphs are actually shown.

The options \hyperlink{config_cfg_dot_graph_max_nodes}{DOT\_\-GRAPH\_\-MAX\_\-NODES} and \hyperlink{config_cfg_max_dot_graph_depth}{MAX\_\-DOT\_\-GRAPH\_\-DEPTH} can be used to limit the size of the various graphs.

The elements in the class diagrams in HTML and RTF have the following meaning: 
\begin{DoxyItemize}
\item A {\bfseries yellow} box indicates a class. A box can have a little marker in the lower right corner to indicate that the class contains base classes that are hidden. For the class diagrams the maximum tree width is currently 8 elements. If a tree is wider some nodes will be hidden. If the box is filled with a dashed pattern the inheritance relation is virtual. 
\item A {\bfseries white} box indicates that the documentation of the class is currently shown. 
\item A {\bfseries grey} box indicates an undocumented class. 
\item A {\bfseries solid dark blue} arrow indicates public inheritance. 
\item A {\bfseries dashed dark green} arrow indicates protected inheritance. 
\item A {\bfseries dotted dark green} arrow indicates private inheritance. 
\end{DoxyItemize}

The elements in the class diagram in $\mbox{\LaTeX}$ have the following meaning: 
\begin{DoxyItemize}
\item A {\bfseries white} box indicates a class. A {\bfseries marker} in the lower right corner of the box indicates that the class has base classes that are hidden. If the box has a {\bfseries dashed} border this indicates virtual inheritance. 
\item A {\bfseries solid} arrow indicates public inheritance. 
\item A {\bfseries dashed} arrow indicates protected inheritance. 
\item A {\bfseries dotted} arrow indicates private inheritance. 
\end{DoxyItemize}

The elements in the graphs generated by the dot tool have the following meaning: 
\begin{DoxyItemize}
\item A {\bfseries white} box indicates a class or struct or file. 
\item A box with a {\bfseries red} border indicates a node that has {\itshape more\/} arrows than are shown! In other words: the graph is {\itshape truncated\/} with respect to this node. The reason why a graph is sometimes truncated is to prevent images from becoming too large. For the graphs generated with dot doxygen tries to limit the width of the resulting image to 1024 pixels. 
\item A {\bfseries black} box indicates that the class' documentation is currently shown. 
\item A {\bfseries dark blue} arrow indicates an include relation (for the include dependency graph) or public inheritance (for the other graphs). 
\item A {\bfseries dark green} arrow indicates protected inheritance. 
\item A {\bfseries dark red} arrow indicates private inheritance. 
\item A {\bfseries purple dashed} arrow indicated a \char`\"{}usage\char`\"{} relation, the edge of the arrow is labled with the variable(s) responsible for the relation. Class {\ttfamily A} uses class {\ttfamily B}, if class {\ttfamily A} has a member variable {\ttfamily m} of type C, where B is a subtype of C (e.g. C could be {\ttfamily B}, {\ttfamily B$\ast$}, {\ttfamily T$<$B$>$$\ast$} ). 
\end{DoxyItemize}

Here are a couple of header files that together show the various diagrams that doxygen can generate:

{\ttfamily diagrams\_\-a.h} 
\begin{DoxyVerbInclude}
#ifndef _DIAGRAMS_A_H
#define _DIAGRAMS_A_H
class A { public: A *m_self; };
#endif
\end{DoxyVerbInclude}
 {\ttfamily diagrams\_\-b.h} 
\begin{DoxyVerbInclude}
#ifndef _DIAGRAMS_B_H
#define _DIAGRAMS_B_H
class A;
class B { public: A *m_a; };
#endif
\end{DoxyVerbInclude}
 {\ttfamily diagrams\_\-c.h} 
\begin{DoxyVerbInclude}
#ifndef _DIAGRAMS_C_H
#define _DIAGRAMS_C_H
#include "diagrams_c.h"
class D;
class C : public A { public: D *m_d; };
#endif
\end{DoxyVerbInclude}
 {\ttfamily diagrams\_\-d.h} 
\begin{DoxyVerbInclude}
#ifndef _DIAGRAM_D_H
#define _DIAGRAM_D_H
#include "diagrams_a.h"
#include "diagrams_b.h"
class C;
class D : virtual protected  A, private B { public: C m_c; };
#endif
\end{DoxyVerbInclude}
 {\ttfamily diagrams\_\-e.h} 
\begin{DoxyVerbInclude}
#ifndef _DIAGRAM_E_H
#define _DIAGRAM_E_H
#include "diagrams_d.h"
class E : public D {};
#endif
\end{DoxyVerbInclude}




 