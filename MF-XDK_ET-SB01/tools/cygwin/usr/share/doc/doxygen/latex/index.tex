\begin{center}  \end{center} 

\subsection*{Introduction}

Doxygen is a documentation system for C++, C, Java, Objective-\/C, Python, IDL (Corba and Microsoft flavors), Fortran, VHDL, PHP, C\#, and to some extent D.

It can help you in three ways: 
\begin{DoxyEnumerate}
\item It can generate an on-\/line documentation browser (in HTML) and/or an off-\/line reference manual (in $\mbox{\LaTeX}$) from a set of documented source files. There is also support for generating output in RTF (MS-\/Word), PostScript, hyperlinked PDF, compressed HTML, and Unix man pages. The documentation is extracted directly from the sources, which makes it much easier to keep the documentation consistent with the source code. 
\item You can \hyperlink{starting_extract_all}{configure} doxygen to extract the code structure from undocumented source files. This is very useful to quickly find your way in large source distributions. You can also visualize the relations between the various elements by means of include dependency graphs, inheritance diagrams, and collaboration diagrams, which are all generated automatically. 
\item You can even `abuse' doxygen for creating normal documentation (as I did for this manual). 
\end{DoxyEnumerate}

Doxygen is developed under \href{http://www.linux.org}{\tt Linux} and Mac OS X, but is set-\/up to be highly portable. As a result, it runs on most other Unix flavors as well. Furthermore, executables for Windows are available.

\par
 This manual is divided into three parts, each of which is divided into several sections.

The first part forms a user manual: 
\begin{DoxyItemize}
\item Section \hyperlink{install}{Installation} discusses how to \href{http://www.doxygen.org/download.html}{\tt download}, compile and install doxygen for your platform. 
\item Section \hyperlink{starting}{Getting started} tells you how to generate your first piece of documentation quickly. 
\item Section \hyperlink{docblocks}{Documenting the code} demonstrates the various ways that code can be documented. 
\item Section \hyperlink{lists}{Lists} show various ways to create lists. 
\item Section \hyperlink{grouping}{Grouping} shows how to group things together. 
\item Section \hyperlink{formulas}{Including formulas} shows how to insert formulas in the documentation. 
\item Section \hyperlink{diagrams}{Graphs and diagrams} describes the diagrams and graphs that doxygen can generate. 
\item Section \hyperlink{preprocessing}{Preprocessing} explains how doxygen deals with macro definitions. 
\item Section \hyperlink{autolink}{Automatic link generation} shows how to put links to files, classes, and members in the documentation. 
\item Section \hyperlink{output}{Output Formats} shows how to generate the various output formats supported by doxygen. 
\item Section \hyperlink{customize}{Customizing the output} explains how you can customize the output generated by doxygen. 
\item Section \hyperlink{custcmd}{Custom Commands} show how to define and use custom commands in your comments. 
\item Section \hyperlink{external}{Linking to external documentation} explains how to let doxygen create links to externally generated documentation. 
\item Section \hyperlink{faq}{Frequently Asked Questions} gives answers to frequently asked questions. 
\item Section \hyperlink{trouble}{Troubleshooting} tells you what to do when you have problems. 
\end{DoxyItemize}

The second part forms a reference manual:


\begin{DoxyItemize}
\item Section \hyperlink{features}{Features} presents an overview of what doxygen can do. 
\item Section \hyperlink{history}{Doxygen History} shows what has changed during the development of doxygen and what still has to be done. 
\item Section \hyperlink{doxygen_usage}{Doxygen usage} shows how to use the {\ttfamily doxygen} program. 
\item Section \hyperlink{doxytag_usage}{Doxytag usage} shows how to use the {\ttfamily doxytag} program. 
\item Section \hyperlink{doxywizard_usage}{Doxywizard usage} shows how to use the {\ttfamily doxywizard} program. 
\item Section \hyperlink{installdox_usage}{Installdox usage} shows how to use the {\ttfamily installdox} script that is generated by doxygen if you use tag files. 
\item Section \hyperlink{config}{Configuration} shows how to fine-\/tune doxygen, so it generates the documentation you want. 
\item Section \hyperlink{commands}{Special Commands} shows an overview of the special commands that can be used within the documentation. 
\item Section \hyperlink{htmlcmds}{HTML Commands} shows an overview of the HTML commands that can be used within the documentation. 
\item Section \hyperlink{xmlcmds}{XML Commands} shows an overview of the C\# style XML commands that can be used within the documentation. 
\end{DoxyItemize}

The third part provides information for developers:


\begin{DoxyItemize}
\item Section \hyperlink{arch}{Doxygen's Internals} gives a global overview of how doxygen is internally structured. 
\item Section \hyperlink{perlmod}{Perl Module output format documentation} shows how to use the PerlMod output. 
\item Section \hyperlink{langhowto}{Internationalization} explains how to add support for new output languages. 
\end{DoxyItemize}

\par
\subsection*{Doxygen license}

\index{license@{license}} \index{GPL@{GPL}}

Copyright \copyright 1997-\/2009 by \href{mailto:dimitri@stack.nl}{\tt Dimitri van Heesch}.

Permission to use, copy, modify, and distribute this software and its documentation under the terms of the GNU General Public License is hereby granted. No representations are made about the suitability of this software for any purpose. It is provided \char`\"{}as is\char`\"{} without express or implied warranty. See the \href{http://www.gnu.org/licenses/old-licenses/gpl-2.0.html}{\tt GNU General Public License} for more details. 

Documents produced by doxygen are derivative works derived from the input used in their production; they are not affected by this license.

\subsection*{User examples}

Doxygen supports a number of \hyperlink{output}{output formats} where HTML is the most popular one. I've gathered   
some nice examples (see {\tt http://www.doxygen.org/results.html})
 of real-\/life projects using doxygen.

These are part of a larger   
list of projects that use doxygen (see {\tt http://www.doxygen.org/projects.html}).
 If you know other projects, let me know and I'll add them.

\subsection*{Future work}

Although doxygen is used successfully by a lot of people already, there is always room for improvement. Therefore, I have compiled a   
todo/wish list (see {\tt http://www.doxygen.org/todo.html})
 of possible and/or requested enhancements.

\subsection*{Acknowledgements}

\index{acknowledgements@{acknowledgements}} Thanks go to: 
\begin{DoxyItemize}
\item \index{Doc++@{Doc++}} Malte Z\"{o}ckler and Roland Wunderling, authors of DOC++. The first version of doxygen borrowed some code of an old version of DOC++. Although I have rewritten practically all code since then, DOC++ has still given me a good start in writing doxygen. 
\item All people at Qt Software, for creating a beautiful GUI Toolkit (which is very useful as a Windows/Unix platform abstraction layer :-\/) 
\item My brother Frank for rendering the logos. 
\item Harm van der Heijden for adding HTML help support. 
\item Wouter Slegers of \href{http://www.yourcreativesolutions.nl}{\tt Your Creative Solutions} for registering the www.doxygen.org domain. 
\item Parker Waechter for adding the RTF output generator. 
\item Joerg Baumann, for adding conditional documentation blocks, PDF links, and the configuration generator. 
\item Tim Mensch for adding the todo command. 
\item Christian Hammond for redesigning the web-\/site. 
\item Ken Wong for providing the HTML tree view code. 
\item Talin for adding support for C\# style comments with XML markup. 
\item Petr Prikryl for coordinating the internationalisation support. All language maintainers for providing translations into many languages. 
\item The band \href{http://www.porcupinetree.com}{\tt Porcupine Tree} for providing hours of great music to listen to while coding. 
\item many, many others for suggestions, patches and bug reports. 
\end{DoxyItemize}